% In this file you should put the actual content of the blueprint.
% It will be used both by the web and the print version.
% It should *not* include the \begin{document}
%
% If you want to split the blueprint content into several files then
% the current file can be a simple sequence of \input. Otherwise It
% can start with a \section or \chapter for instance.

% In this file you should put the actual content of the blueprint.
% It will be used both by the web and the print version.
% It should *not* include the \begin{document}
%
% If you want to split the blueprint content into several files then
% the current file can be a simple sequence of \input. Otherwise It
% can start with a \section or \chapter for instance.

\begin{abstract}
    We outline a plan of formalization for Kripke frames, define the modal operators $\square$, $\diamond$ and characterise some of the most relevant modal logic axioms. Time permitting, we will formalize some examples.
  \end{abstract}
  
  
  This blueprint, and more generally the \href{https://github.com/MatejJazbec/kripke-frames}{\texttt{kripke-frames}} Lean project, is part of the assignment for the course \href{https://www.andrej.com/zapiski/MAT-FORMATH-2024/book/}{Formalized mathematics and proof assistants}.

  
  \section{Basic definitions}
  
  A \emph{Kripke frame} or \emph{modal frame} is a pair $\langle W, R \rangle$ where $W$ is a (possibly empty) set, and $R$ is a \textbf{binary relation} on $W$. Elements of $W$ are called \emph{nodes} or \emph{worlds}, and $R$ is known as the \textbf{accessibility relation}.

  \begin{definition}
    \label{def:kripke-frame}
    %
    \lean{KripkeFrame}
    \leanok
    %
    A \emph{Kripke model} is a triple $\langle W, R, \Vdash \rangle$, where $\langle W, R \rangle$ is a Kripke frame, and $\Vdash$ is a relation between nodes of $W$ and modal formulas,
     such that for all $w \in W$ and modal formulas $A$ and $B$:

    \begin{itemize}
        \item $w \Vdash \neg A$ if and only if $w \nVdash A$,
        \item $w \Vdash A \to B$ if and only if $w \nVdash A$ or $w \Vdash B$,
        \item $w \Vdash \Box A$ if and only if $u \Vdash A$ for all $u$ such that $w \; R \; u$.
    \end{itemize}.
 \end{definition}

 We read $w \Vdash A$ as “$w$ satisfies $A$”, “$A$ is satisfied in $w$”, or “$w$ forces $A$”. The relation $\Vdash$ is called the \emph{satisfaction relation},
  \emph{evaluation}, or \emph{forcing relation}. The satisfaction relation is uniquely determined by its value on propositional variables.

A formula $A$ is \emph{valid} in:
\begin{itemize}
    \item a model $\langle W, R, \Vdash \rangle$, if $w \Vdash A$ for all $w \in W$,
    \item a frame $\langle W, R \rangle$, if it is valid in $\langle W, R, \Vdash \rangle$ for all possible choices of $\Vdash$,
    \item a class $C$ of frames or models, if it is valid in every member of $C$.
\end{itemize}



Consider the schema \( T : \Box A \to A \). \( T \) is valid in any reflexive frame \( \langle W, R \rangle \): if \( w \Vdash \Box A \), then \( w \Vdash A \) since \( w \, R \, w \).
 On the other hand, a frame which validates \( T \) has to be reflexive: fix \( w \in W \), and define satisfaction of a propositional variable \( p \) as follows: \( u \Vdash p \)
  if and only if \( w \, R \, u \). Then \( w \Vdash \Box p \), thus \( w \Vdash p \) by \( T \), which means \( w \, R \, w \) using the definition of \( \Vdash \). \( T \) corresponds
   to the class of reflexive Kripke frames.





\section*{Common Modal Axiom Schemata}

The following table lists common modal axioms together with their corresponding classes.
\[
\begin{array}{|c|c|c|}
\hline
\textbf{Name} & \textbf{Axiom} & \textbf{Frame Condition} \\
\hline
\textbf{K} & \Box (A \to B) \to (\Box A \to \Box B) & \text{Holds true for any frames} \\
\hline
\textbf{T} & \Box A \to A & \text{Reflexive: } w\,R\,w \\
\hline
- & \Box\Box A \to \Box A & \text{Dense: } w\,R\,u \Rightarrow \exists v\,(w\,R\,v \land v\,R\,u) \\
\hline
\textbf{4} & \Box A \to \Box\Box A & \text{Transitive: } w\,R\,v \wedge v\,R\,u \Rightarrow w\,R\,u \\
\hline
\textbf{D} & \Box A \to \Diamond A \text{ or } \Diamond\top \text{ or } \neg\Box\bot & \text{Serial: } \forall w\,\exists v\,(w\,R\,v) \\
\hline
\textbf{B} & A \to \Box\Diamond A \text{ or } \Diamond\Box A \to A & \text{Symmetric: } w\,R\,v \Rightarrow v\,R\,w \\
\hline
\textbf{5} & \Diamond A \to \Box\Diamond A & \text{Euclidean: } w\,R\,u \land w\,R\,v \Rightarrow u\,R\,v \\
\hline
\end{array}
\]

\section*{Examples}
Time permitting, we will formalize some examples. This section will thus be extended in due time.


 
  
  %%% Local Variables:
  %%% mode: LaTeX
  %%% TeX-master: "print"
  %%% End:

